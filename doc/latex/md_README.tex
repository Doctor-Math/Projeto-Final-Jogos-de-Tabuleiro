\section*{Projeto Final -\/ Jogos de Tabuleiro ♟ }

{\bfseries Aplicar os conceitos de P\+D\+S2 em um programa funcional que reflete nosso aprendizado durante as aulas}


\begin{DoxyItemize}
\item Participantes 🧑‍💻\+: Guilherme Bueno de Andrade Motta, Gustavo Cabral Gonçalves, Matheus Soares e Renato Lucas
\end{DoxyItemize}

{\bfseries Introdução}

Neste projeto, vamos aplicar técnicas aprendidas em sala de aula com o objetivo de criar um programa que reproduz o funcionamento de vários jogos tabuleiro, sendo eles Jogo da Velha, Lig4 e Reversi. Neste programa foram aplicados conceitos de Modularização, Classes, Programação Orientada a Objetos (P\+OO), entre outros. Para realizar o projeto utilizamos das ferramentas do Git\+Hub, V\+Scode, Doxygen e Docteste. Tal programa foi feito com a finalidade para que possamos simular um ambiente de programação em equipe e também nos ambientarmos com as novas técnicas aprendidas.\hypertarget{md_README_autotoc_md0}{}\section{Projeto 🏆}\label{md_README_autotoc_md0}
{\bfseries Modularização 🐉}

O programa é dividido em módulos para que ele seja melhor organizado fácilitando a entende-\/lo e também para ajudar o trabalho em equipe para que vários participantes possam programas em diferentes módulos ao mesmo tempo.

A modularização foi dividida em pastas. Os arquivos hpp, que estão na pasta include, Definem as classes e funções utilizadas no projeto. Já os arquivos com as implementações, os cpp, estão na pasta src, eles definem as classes e funções utilizadas no projeto para serem utilizadas no arquivo main. Além deles temos o arquivo principal(main) e o Makefile, o primeiro deles é responsável por executar o programa e o segundo é encarregado de compilar o projeto automaticamente.

{\itshape {\bfseries C\+L\+A\+S\+S\+E\+S🥋}}

{\bfseries 🟣\+Classe Abstrata}\+: Jogo\+Tabuleiro

🧐\+Responsabilidades\+:

Fornecer a estrutura do tabuleiro;

Fornecer peças;

Ler uma jogada;

Testar a validade da jogada;

Testar condições de vitória;

Imprimir tabuleiro.

🤝\+Colaboração\+:

Nenhuma.

{\bfseries 🟠\+Classe Herdeira}\+: Jogo\+Da\+Velha

🧐\+Responsabilidades\+:

(Herdada de Jogo\+Tabuleiro) Fornecer a estrutura do tabuleiro;

(Herdada de Jogo\+Tabuleiro) Fornecer peças;

(Herdada de Jogo\+Tabuleiro) Ler uma jogada;

(Herdada de Jogo\+Tabuleiro) Sobrescreve o método validar\+Jogada() para testar a validade da jogada no jogo da velha;

(Herdada de Jogo\+Tabuleiro) Sobrescreve o método verificar\+Vitoria() para testar as condições de vitória específicas do jogo da velha;

(Herdada de Jogo\+Tabuleiro) Sobrescreve o método imprimir\+Tabuleiro() para imprimir o tabuleiro.\+ 

🤝\+Colaboração\+:

Jogo\+Tabuleiro.

{\bfseries 🟠\+Classe Herdeira}\+: Lig4

🧐\+Responsabilidades\+:

(Herdada de Jogo\+Tabuleiro) Fornecer a estrutura do tabuleiro;

(Herdada de Jogo\+Tabuleiro) Fornecer peças;

(Herdada de Jogo\+Tabuleiro) Ler uma jogada;

(Herdada de Jogo\+Tabuleiro) Sobrescreve o método validar\+Jogada() para testar a validade da jogada no jogo lig4;

(Herdada de Jogo\+Tabuleiro) Sobrescreve o método verificar\+Vitoria() para testar as condições de vitória específicas do lig4;

(Herdada de Jogo\+Tabuleiro) Sobrescreve o método imprimir\+Tabuleiro() para imprimir o tabuleiro.\+ 

🤝\+Colaboração\+:

Jogo\+Tabuleiro.

{\bfseries 🟠\+Classe Herdeira}\+: Reversi

🧐\+Responsabilidades\+:

(Herdado de Jogo\+Tabuleiro) Fornecer a estrutura do tabuleiro.

(Herdado de Jogo\+Tabuleiro) Fornecer peças.

(Herdado de Jogo\+Tabuleiro) Ler uma jogada.

(Herdado de Jogo\+Tabuleiro) Sobrescreve o método validar\+Jogada() para testar a validade da jogada no jogo Reversi.

(Herdado de Jogo\+Tabuleiro) Sobrescreve o método verificar\+Vitoria() para verificar as condições de vitória específicas do Reversi.

(Herdado de Jogo\+Tabuleiro) Sobrescreve o método imprimir\+Tabuleiro() para imprimir o tabuleiro

Novo método verificar\+Direcao(int, int, int, int, int)\+: Método que verifica se há peças alinhadas em uma determinada direção (útil para capturar peças no Reversi).

🤝\+Colaboração\+:

Jogo\+Tabuleiro.\hypertarget{md_README_autotoc_md1}{}\section{Funcionamento Do Programa 🧑‍💻}\label{md_README_autotoc_md1}
No início do programa o usuário pode se cadastrar escolhendo o seu nome e apelido, esses dados vão para um arquivo no qual o usuário pode cadastrar mais jogadores ou remover os já criados. Após o cadastro, o usuário escolhe qual jogo ele quer iniciar a depender das letras que ele digitar, por exemplo R -\/$>$ Reversi, L -\/$>$Lig e V-\/$>$ Jogo da Velha. Na execução das partidas o usuário deve entrar com os apelidos dos dois jogadores e o sistema mostrar uma partida do jogo selecionado, sendo ela interatica na qual os dois jogadores podem competir entre si. Durante a partida o sistema ele é capaz de checar se uma jogada é válida ou não, atualizar o jogo a cada jogada e testar uma possível vitória de um jogador. No final do jogo, o programa mostra o vencedor e no banco de dados que contém os nomes e apelidos dos jogadores ele atualiza as estatísticas dos indivíduos.\hypertarget{md_README_autotoc_md2}{}\section{Dificuldades 😮‍💨}\label{md_README_autotoc_md2}
A criação desse algoritmo foi um processo ao mesmo tempo muito divertido, foi também custoso. A liberdade de poder criar um código da nossa maneira, poder usar todos os nossos conhecimentos e quando não sabiamos o que fazer, pesquisar na internet, conversar entre nós e com outros colegas sobre alguma solução para o problema, foram situações satisfatórias.

Em muitos momentos, pensavamos que o código estava funcionando da maneira correta e ao prosseguir por outro teste e ele dar totalmente errado era bem desagradável, mas sempre uma oportunidade de procurar o erro e resolve-\/lo da melhor maneira possível.

😡\+Nossas principais dificuldades foram\+:


\begin{DoxyItemize}
\item Conseguir fazer a verificação de vitórias no jogo Reversi e Lig4
\item Se ambientar com o Git\+Hub e suas funcionalidades
\item Conseguir que o sistema conseguisse entender e marcar cada jogada
\item Aprender o usar o doxygen
\item Tentar entender a lógica geral do programa 
\end{DoxyItemize}